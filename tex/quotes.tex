

``Life is like riding a bicycle. To keep your balance you must keep moving.''

Albert Einstein as quoted in Walter Isaacson, \emph{Einstein: His Life and Universe} (2007), p. 367


``I believe in intuition and inspiration. Imagination is more important than knowledge. For knowledge is limited, whereas imagination embraces the entire world, stimulating progress, giving birth to evolution. It is, strictly speaking, a real factor in scientific research.''

Albert Einsten in \emph{Cosmic Religion : With Other Opinions and Aphorisms} (1931), p. 97


``Physical concepts are free creations of the human mind, and are not, however it may seem, uniquely determined by the external world. In our endeavor to understand reality we are somewhat like a man trying to understand the mechanism of a closed watch. He sees the face and the moving hands, even hears its ticking, but he has no way of opening the case. If he is ingenious he may form some picture of a mechanism which could be responsible for all the things he observes, but he may never be quite sure his picture is the only one which could explain his observations.''

Albert Einstein in \emph{The Evolution of Physics} (1938)


``I have no special talents. I am only passionately curious.''

Albert Einstein in Letter to Carl Seelig (11 March 1952)


``The important thing is not to stop questioning; curiosity has its own reason for existing. One cannot help but be in awe when contemplating the mysteries of eternity, of life, of the marvelous structure of reality. It is enough if one tries merely to comprehend a little of the mystery every day. The important thing is not to stop questioning; never lose a holy curiosity.''

Albert Einstein as quoted in LIFE magazine (2 May 1955)


``When a man sits with a pretty girl for an hour, it seems like a minute. But let him sit on a hot stove for a minute and it's longer than any hour. That's relativity.''

quote from, actually, the abstract from a short paper written by Einstein that appeared in the now defunct Journal of Exothermic Science and Technology (JEST, Vol. 1, No. 9; 1938).


``When you meet someone better than yourself, turn your thoughts to becoming his equal. When you meet someone not as good as you are, look within and examine your own self.''

Confucius in \emph{The Analects} (cca.\@ 475 BCE -- 221 BCE)



``No. Scientists do not compromise. Our minds are trained to synthesize facts and come to inarguable conclusions. Not to mention Sheldon is bat-crap crazy.''

Leonard responding to Penny's proposal to make peace with Sheldon; \emph{the Big Bang Theory} created by Chuck Lorre and Bill Prady.


``Looking out at your fresh young faces, I remember when I, too, was deciding my academic future as a lowly graduate student. Of course, I was 14, and I had already achieved more than most of you could ever hope to despite my 9 o'clock bedtime. Now, there may be one or two of you in this room who has what it takes to succeed in theoretical physics, although it's more likely that you'll spend your scientific careers teaching fifth graders how to make papier-m\^{a}ch\'{e} volcanoes with baking soda lava.''

Leonard in \emph{the Big Bang Theory} created by Chuck Lorre and Bill Prady.


``One cannot really be considered as having a research topic until it can be expressed in the form of a succinct question''

``Keep in mind the three Cs of research --- curiosity, concentration and confidence''

``Where subjective judgements are involved concerning the value of work, it is possible that the same filter prevents some useful material from being published, or being published in the most appropriate form''

New discoveries include \citep{Creedy2008EE}:
\begin{itemize}
  \item new empirical regularities,
  \item new theoretical results, and/or
  \item improved understanding of and fresh insights into a problem;
\end{itemize}

Sometimes, research projects can lead to `\emph{negative}' results --- proving that the proposed methodological or technological improvement does not fit expectations or does not help solve some practical problem. Many agree (e.g.\@ \citet{Creedy2008EE}) that even such disappointing results can be useful and should not be dismissed. In fact, some of the best articles in the history of science focused on proving that something DOES NOT WORK! ** any good examples?! **

``Progress in research is largely achieved by making a series of small steps, rather than taking giant leaps''

``Progress in research is actually highly nonlinear. Papers are often completed as a result of the pressure of deadlines, or the need to turn to other work, rather than ending in a dramatic flourish$\ldots$ the completion of a research paper is therefore often accompanied by negative feelings that after all, not much has been achieved''

``Do not fall in love with your own writing.''

``Perhaps the most important rule of writing is that the first draft is not the final draft but is simply the start of a long process of revision.''

``Starting a PhD thesis is typically a leap in the dark. This naturally leads to anxieties.''


``The last thing one knows in constructing a work is what to put in first.''

``Thinking too little about things or thinking too much both make us obstinate and fanatical.''

``It is man's natural sickness to believe that he possesses the Truth.''

``Clarity of mind means clarity of passion, too.''

Blaise Pascal


``The combination of some data and an aching desire for an answer does not ensure that a reasonable answer can be extracted from a given body of data''

John Tukey in


``The way to create art is to burn and destroy ordinary concepts and to substitute them with new truths that run down from the top of the head and out from the hearth''

Charles Bukowski in ``Sifting Through the Madness for the Word, the Line, the Way'' (2003)


``Politics and religion are both obsolete; the time has come for science and spirituality''

Arthur C.\@ Clarke quoting Nehru on the discussion about religion


``Keys to success in scientific writing are:''
\begin{itemize}
  \item deep reflection
  \item talk to people
  \item use mind-altering devices
  \item no ipod!
\end{itemize}

Alex McBratney's talk at the Pedometrics 2007 conference


``Talent imitates, genius steals!''

Edward Tufte at the one-day workshop on presenting data and information, 28 March 2010, Pittsburg (PA)



